\chapter{Lecture 2: Nonlinear Dynamics}\label{lecture2}
\section{The Simple Pendulum}
Even the most simple dynamical systems, e.g. the simple pendulum, can not be solved in a closed form. This is due to the nonlinear characteristics of the underlaying differential equations. But actually you don't have to in order to describe and analyse fundamental dynamical characteristics. 

But what we really care about is the long-term behaviour of the sytem. For low dimensional systems, two central tools are available in order to analyse the systems behaviour: 
\begin{itemize}
\item Linearization
\item Graphical Analysis
\end{itemize}
 
The equations of motion of the simple pendulum can be derived with the Langrangian as:
$$ ml^2 \ddot\theta(t) + mgl\sin{\theta(t)} = Q $$ 
Considering the generalized force $Q$ as combination of damping and a control torque input
$$ Q = -b\dot\theta(t) +u(t).$$
For the case of a constant torque this yields
$$ ml^2 \ddot\theta + b\dot\theta + mgl \sin\theta = u_0.$$

\section{Graphical Analysis}
\subsection{Fixed Points}
The central goal of control is to meaninfully shift the vector field via sophisticated control input of a system in order to change its dynamics.  
So called \textit{phase plots} are useful for visualizing this vector field of two-dimensional systems.  In case of the state vector $x= \begin{bmatrix} \theta \\ \dot{\theta} \end{bmatrix}$ this means $\dot{\theta}$ over $\theta$.

\begin{definition}
A point the system will remain forever without applying external forces is called a fixed point or a steady state respectively.
\end{definition}
The position of fixed points, e.g. stable positions of the pendulum, strongly depend on the parameter of the system (damping, input torque etc.). 

\subsection{Definitions of Stability}
There are existing different types of stability in order to describe the behaviour next a fixed point $x^{\star}$. The fixed point can be 
\begin{itemize}
\item \textit{Stable} in the sense of Lyapunov (i.e. will remain within certain radius)
\item \textit{Asymptotically stable} (i.e. for $t->\infty$ reaches certain point)
\item \textit{Exponentially stable} (reaches certain point at defined rate).
\end{itemize}
