\chapter{Lecture 19: Model Systems with Stochasticity}
So far: 
$$\dot{\myM{x}}=f(\myM{x},\myM{u}).$$
Now: 
$$\dot{\myM{x}}=f(\myM{x},\myM{u},\myM{w}),$$
where $\myM{w}$ is the output of some random process. Here we only cover \textbf{process noise}, caused by:
\begin{itemize}
\item Disturbances
\item Model Uncertainty
\end{itemize}
Measurement noise will not be considered here.

Every time that we've analyzed a system to date, we've asked questions like "given x[0], what is the long-term behavior of the system? But now $x[n]$ is a random variable. The trajectories of this system do not converge, and the system does not exhibit any form of stability that we've introduced so far. 


\section{The Master Equation}
All is not lost. The \textbf{distribution} of this random variable is actually very well behaved. This is the key idea for this chapter. Let us use $p_n(x)$ to denote the probability density function over the random variable $\myM{x}$ at time $n$.

It is actually possible to write the dynamics of the probability density with the simple relation 
$$p_{n+1}(x) = \int_{-\infty}^\infty p(x|x') p_n(x') dx',$$
where $p(x|x')$ encodes the stochastic dynamics as a conditional distribution of the next state (here x) as a function of the current state (here $x'$). Dynamical systems that can be encoded in this way are known as continuous-state Markov Processes, and the governing equation above is often referred to as the "master equation" for the stochastic process.


\section{When Do We Actually Need This Kind of Modeling?}
Let's answer this question by two different kinds of disturbances that can be applied to legged locomotion: 
\begin{itemize}
\item \textbf{The robot gets kicked:} One huge impact, but the model is correct for most of the time (after recovery from push and before being kicked
\item \textbf{The robot is walking on uneven terrain:} The model of flat terrain is wrong for all the time. Here it makes sense to add stochasticity to the modeling approach.
\end{itemize}