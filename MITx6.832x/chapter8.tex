\chapter{Lecture 14/15: Computational Tools for Legged Robots i.e.: Planning and Control through Contact}
So far we've discovered the following tools:
\begin{itemize}
\item Fixed Points / Local Stability
\item Local Stabilization (e.g. LQR)
\item Lyapunov Analysis
\item Trajectory Optimization
\end{itemize}
These tools work out for "smooth" systems where the equations of motion are described by a function $\dot{\myM{x}}=f(\myM{x},\myM{u})$ which is smooth everywhere. But our discussion of the simple models of legged robots illustrated that the dynamics of making and breaking contact with the world are more complex -- these are often modeled as \textbf{hybrid dynamics} with \textbf{impact} discontinuities at the collision event \textbf{and constrained dynamics} during contact (with either soft or hard constraints).

The goal of this chapter is to extend our computational tools into this richer class of models. Many of our core tools still work: trajectory optimization, Lyapunov analysis (e.g. with sums-of-squares), and LQR all have natural equivalents. 

\section{Modeling Contacts}
We can model the robot in \textbf{floating-base coordinates} -- we add a fictitious six degree-of-freedom "floating-base" joint connecting some part of the robot to the world (in planar models, we use just three degrees-of-freedom. We can derive the equations of motion for the floating-base robot once, without considering contact, then add the additional constraints that come from being in contact as contact forces which get applied to the bodies. The resulting manipulator equations take the form
$$\myM{M}(\myM{q})\ddot{\myM{q}}+\myM{C}(\myM{q,\dot{q}})\dot{\myM{q}}=\tau_g(\myM{q})+\myM{Bu}+\sum_i\myM{J}_i^T(\myM{q})\lambda_i,$$
where $\lambda_i$ are the contact forces and $J_i$ are the contact Jacobians. 
Conveniently, if the guard function in our contact equations is the signed distance from contact, $\phi_i(\myM{q})$, then this Jacobian is simply $\myM{J}_i(\myM{q})=\dfrac{\partial\phi_i}{\partial\myM{q}}$.

\section{Trajectory Optimization}
\section{Randomized Motion Planning}
\section{Stabilizing a Trajectory or Limit Cycle}


