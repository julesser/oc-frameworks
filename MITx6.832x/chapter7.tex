\chapter{Lecture 12/13: Simple Models of Walking and Running}
In this chapter we'll introduce some of the simple models of walking and robots, the control problems that result, and a very brief summary of some of the control solutions described in the literature. Compared to the robots that we have studied so far, our investigations of legged locomotion will require additional tools for thinking about limit cycle dynamics and dealing with impacts.


\section{Limit Cycles}
In many of the systems that we have studied so far, we have analyzed the stability of a fixed-point, or even an (infinite-horizon) trajectory. For walking systems the natural equivalent is to talk about the stability of periodic solutions -- a fixed "gait" is a cycle that repeats footstep after footstep. So we begin our discussion with a discussion of the stability of a cycle.
A limit cycle is an asymptotically stable or unstable periodic orbit. One of the simplest models of limit cycle behavior is the Van der Pol oscillator.
\subsection{Poincare Maps}


\section{Simple Models of Walking}
\subsection{The Rimless Wheel}
The most elementary model of passive dynamic walking, first used in the context of walking by, is the rimless wheel. This simplified system has rigid legs and only a point mass at the hip as illustrated in the figure above. To further simplify the analysis, we make the following modeling assumptions
\begin{itemize}
\item No slip
\item Collisions are inelastic and impulsive (no bouncing)
\item No double support
\end{itemize}
\subsection{The Compass Gait}
\subsection{The Kneed Walker}


\section{Simple Models of Running}
\subsection{The Spring-Loaded Inverted Pendulum}
\subsection{The Planar Monopod Hopper}


\section{A Simple Model That Can Walk and Run}