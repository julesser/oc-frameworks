\chapter{DRAKE - MIT CSAIL}
\section{Introduction}
\subsection{Motivation}
Drake \cite{drake} is a \textbf{C++ toolbox for}
\begin{itemize}
\item Analyzing the dynamics of robots
\item Building control systems for robots
\item Heavy emphasis on optimization-based design/analysis
\end{itemize}

Drake aims to \textbf{simulate} 
\begin{itemize}
\item Complex dynamics of robots (e.g. including friction, contact, aerodynamics etc.)
\item Emphasis on exposing the structure in the governing equations (sparsity, analytical gradients, polynomial structure, uncertainty etc.)
\item Making this information available for advanced planning, control, and analysis algorithms
\end{itemize}

Drake \textbf{provides}
\begin{itemize}
\item Python Interface
\item Implementation of state-of-the-art algorithms
\item Various examples
\end{itemize}

\subsection{Core Modules}
Drakes functionality is incorporated within several modules. This section gives a brief overview. 
\subsubsection{Modeling Dynamical Systems}
Drake uses a Simulink-inspired description of dynamical systems.

Includes basic building blocks (adders, integrators, delays, etc), physics models of mechanical systems, and a growing list of sensors, actuators, controllers, planners, estimators.
\subsubsection{Solving Mathematical Programs}
Drake's MathematicalProgram class is used to solve the mathematical optimization problem in the following form
$$min_{x} f(x) \quad s.t. x \in S$$
Depending on the formulation of the objective function $f$, and the structure of the constraint set $S$, Drake can solve the following categories of optimization problems
\begin{itemize}
\item Linear programming
\item Quadratic programming 
\item Nonlinear nonconvex programming
\item Semidefinite programming
\item Sum-of-squares programming
\item Mixed-integer programming
\end{itemize}
Drake \textbf{automatically} calls suitable solvers for each category of optimization problem.
\subsubsection{Multibody Kinematics and Dynamics}
\begin{itemize}
\item Drake's \textbf{constraint system} helps solve computational dynamics problems with algebraic constraints
\item Drake approximates real-world physical \textbf{contact} phenomena with a combination of geometric techniques and response models.
\end{itemize}

\section{How-To}
\subsection{Install}
\subsection{Run Examples}


\section{Working with the Examples}
\subsection{Double Pendulum}
\subsection{Cart-Pole}
\subsection{Rimless Wheel}
\subsection{Compass Gait}
\subsection{Spring-Loaded Inverted Pendulum}

