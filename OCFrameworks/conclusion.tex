\chapter{CONCLUSION}

Within this two-month student project it was possible to integrate the RH5 robot in the \textbf{Crocoddyl} framework and generate first successfull trajectories for a simple walk (see chapter \ref{sec:resultsRH5}). These preliminary results offer much space for improvement and future research opportunities.

The investigation of the \textbf{Drake} library was limited to simple passive dynamic walking models (see \ref{sec:drakeExamples}). An estimation of producing comparable bipedal walking results like in Crocoddyl within the Drake framework, revealed a workload of approximitive 40-60 hours (see section \ref{sec:BipedWalkCapabilities} for details). 

In the end, the author considers \textbf{both framworks suitable for further research} and application to the RH5 robot. 
Whereas Drake offers a more general framework for optimization-based analysis (various solvers, planner, controller etc.) and does not provide targeted examples for bipedal walking of highly-articulated robots, Crocoddyl focuses exactly on these kind of problems, offers great example functionalities, but also limits the available solvers to the special branch of DDP-based solvers.